\documentclass[twoside, 12pt]{iiser-thesis}

%%%%%%%%%%%%%%%%%%%
% Packages/Macros %
%%%%%%%%%%%%%%%%%%%

\usepackage{fullpage}

\usepackage{amssymb,latexsym,amsmath}     % Standard packages
\usepackage{graphicx}
\usepackage{authblk}
\usepackage{dsfont}
\usepackage{amsthm}
\usepackage{color}
\usepackage{float}
\usepackage{tikz}
\usepackage{tikz-cd}
%\usepackage{hyperref}
\setlength{\parskip}{1em}
\usepackage{hyperref}
\usepackage[capitalize]{cleveref}
%
% please place your own definitions here and don't use \def but
% THEOREM Environments ---------------------------------------------------

\newtheorem{thm}{Theorem}[section]
\newtheorem{prop}[thm]{ Proposition}
\newtheorem{lem}[thm]{ Lemma}
\newtheorem{cor}[thm]{ Corollary}
\newtheorem{ex}{ Example}
\newtheorem{exer}{ Exercise}
\newtheorem{rem}{Remark}[section]
\newtheorem{rems}[rem]{ Remarks}
\newtheorem{remark}[rem]{Remark}
\newtheorem{defi}{ Definition}[section]
\newtheorem{defn1}[defi]{Definition}
\newtheorem{defs}[defi]{Definitions}
\newtheorem{notation}{ Notation}[section]
%\newtheorem{mypar}{{\bf }}[section]
%\newcommand{\skp}{\vspace{\baselineskip}}
%%\newcommand{\qed}{\hfill\rule{1.6mm}{1.6mm}}
%\renewcommand{\proof}{\noindent{\bf Proof.\ }}
%\newcommand{\no}{\nonumber}
%\newcommand{\noi}{\noindent}
%\newcommand{\txt}{\textrm}
%\newcommand{\pa}{\partial}
%\newcommand{\ds}{\displaystyle}
\newcommand{\R}{\mathbb{R}}
\newcommand{\Z}{\mathbb{Z}}
\newcommand{\C}{\mathbb{C}}
\newcommand{\Q}{\mathbb{Q}}
\newcommand{\A}{\mathbb{A}}
\newcommand{\I}{\mathbb{I}}
\newcommand{\Ol}{\mathcal{O}}
\newcommand{\gal}{\text{Gal}}
\newcommand{\x}{\times}
\newcommand{\un}{\text{unr}}
\newcommand{\s}{\text{sep}}
\newcommand{\al}{\text{al}}
\newcommand{\ab}{\text{ab}}
%\newcommand{\si}{\sigma}
%\newcommand{\al}{\alpha}
%\newcommand{\la}{\lambda}
%\newcommand{\La}{\Lambda}
%\newcommand{\calF}{\mathcal{F}}
%\newcommand{\eps}{\varepsilon}
%\newcommand{\ph}{\varphi}
%\newcommand{\sig}{\sigma}
%\newcommand{\tab}{\hspace*{0.3in}}
%\newcommand{\Tab}{\hspace*{1.0in}}
%\newcommand{\vf}{\varphi}
%\newcommand{\del}{\frac{\partial}{\partial t}}
%\newcommand{\vp}{\varepsilon}


\catcode`\@=11
\catcode`\@=12


%%%%%%%%%%%%
% Document %
%%%%%%%%%%%%
%\setcounter{chapter}{-1}

\title{Class Field Theory and Introduction to Iwasawa Theory}


\author{Ashutosh Jangle}
\coordinator{Coordinator} \supervisor{Name of your Guide} \sdesignation{ ? Professor}
\department{Department of Mathematics} \reader{Name of Your TAC}
%\reader{Reader 2}
\dedication{This thesis is dedicated to ?}
\graduationyear{2018}
\academicyear{2017-2018}
\graduationmonth{April}

\thesisabstract{ Write your  abstract here
}
\acknowledgments{Not more than 250 words}

\begin{document}
	\thesisfront
	

\chapter{Preliminaries}
\begin{lem}\label{coeffbound}
Let $n,N \geq 1$ , the number of algebraic integers $\alpha$, such that $[ \Q [\alpha ]:\Q ] \leq n $ and $|\sigma (\alpha ) | \leq N$ for all archimedean embeddings $\sigma$ is finite.
\end{lem}
\begin{proof}
Let $\min (\alpha , \Z )= \prod _\sigma (x-\sigma (\alpha ))$. Coefficients of this minimal polynomial are bounded as $|a_i | \leq N^{n-1} {n \choose i}$. 
\end{proof} 




\section{Dedekind domains}

\begin{notation}
    Let $R$ be a Dedekind domain, $K$ be its quotient field, $U$ a vector space over $K$, $T$ a submodule of $U$. $T_\mathfrak p = R_\mathfrak p T$. $L,M,N$ denote finitely generated $R$ submodules which span $U$.
\end{notation}
\begin{prop}
$\cap _\mathfrak p T_\mathfrak p = T$.
\end{prop}	
\begin{proof}
This holds true for any integral domain in fact as long as we intersect over all the maximal ideals.
\newline 
Obviously, $T \subset \cap _\mathfrak p T_\mathfrak p$. For the other direction, for $u\in \cap _\mathfrak p T_\mathfrak p$ let $$ J_u = \{x \in R \ \big | \  xu \in T \}$$ But $u=x^{-1}w$ for some $w\in T$ and $x \in R$ and $x\notin \mathfrak p$. Thus $J_u \not \subset \mathfrak p$. As this is the case for every $\mathfrak p$, $J_u=R$. 
\end{proof}
\begin{lem}
Given $M$ and $N$ there is nonzero $a\in R$ such that $aM \subset N$.
\end{lem}
\begin{lem}
For all but finitely many $\mathfrak p$, $M_\mathfrak p = N_\mathfrak p$.
\end{lem}

\subsection{Discriminants}
\begin{defi}
Let $L/K$ be a finite extension and define symmetric bilinear form $$(\alpha, \beta ) \mapsto Tr(\alpha, \beta)$$The discriinant of this form is called the discriminant of $L/K$. 
 \newline More generally given rings $A \subset B$, given $\beta _1, \dots , \beta _n$ elements of $B$, their discriminant is $$D(\beta _1, \dots , \beta _n) = \det (Tr(\beta _i \beta _j))$$ 
\end{defi}
\begin{lem}
If $\gamma _j = \sum _i a_{ij}\beta _i$ for $a_{ij}\in A$ then $$D(\gamma _1, \dots , \gamma _n) = \det (a_{ij})^2 D(\beta _1, \dots , \beta _n).$$
\end{lem}
\section{Local Fields}
\begin{prop}
The completion of discrete valuation ring $R$ is canonically isomorphic to $\varprojlim R/ \mathfrak m ^n$ where $\mathfrak m$ is the maximal ideal of $R$.
\end{prop}
\chapter{Group Cohomology}
\begin{lem}[Snake lemma]
\end{lem}
\begin{prop}\label{inverselimitexact}
Let $\{A_i\},\{B_i\}, \{C_i\}$ be inverse system of abelian groups and \[ \begin{tikzcd}
0 \arrow[r] & A_i \arrow[r] & B_i \arrow[r] & C_i \arrow[r] & 0 
\end{tikzcd}\] be exact for all $i$ such that 
\[ \begin{tikzcd}
 0 \arrow [r] & A_{i+1} \arrow[r] \arrow[d] & B_{i+1} \arrow[d] \arrow[r] & C_{i+1} \arrow[r] \arrow[d] & 0 \\
0 \arrow[r] & A_i \arrow[r] & B_i \arrow[r] & C_i \arrow[r] & 0
\end{tikzcd} \] then we find that 
\[ \begin{tikzcd}
0 \arrow[r] & \varprojlim A_i \arrow[r] & \varprojlim B_i \arrow[r] & \varprojlim C_i 
\end{tikzcd}\]
is exact and if all $\theta_{i+1}:A_{i+1}\rightarrow A_i$ are surjective then \[ \begin{tikzcd}
0 \arrow[r] & \varprojlim A_i \arrow[r] & \varprojlim B_i \arrow[r] & \varprojlim C_i \arrow[r] & 0
\end{tikzcd}\] is exact.
\end{prop}
\begin{proof}
See \cite{atiyah} Proposition 10.2.
\end{proof}

\begin{thm}[Shapiro's lemma]\label{shapiro}
Let $G$ be a finite group, let $H$ be a subgroup and let $B$ be an $H$-module. Then for every $i\in \Z$ we have canonical isomorphisms \[ \hat H^i (G, \text{CoInd}^G_H(B)) \cong \hat H^i(H,B)\] that together provide natural isomorphisms of $\delta$-functors.
\end{thm}
\begin{prop}
Suppose that $H$ is a subgroup of finite index in $G$ and $B$ is a $H$-module. Then we have a cononical isomorphism of $G$-modules \[\chi: \text{CoInd}^G_H(B) \xrightarrow{\sim}\text{Ind}^G_H(B), \quad \chi (\phi) = \sum _{\overline g \in H \backslash G } g^{-1}\otimes \phi (g)\]
where for each $\overline g \in H \backslash G$ the element $g\in G$ is an arbitrary choice of representative of $\overline g$.
\end{prop}
\begin{proof}
See \cite{sharifi} Proposition 7.5.4.
\end{proof}



\subsection{Herbrand quotient}
\begin{lem}\label{herbrandtensorq}
Let $G$ be finite, cyclic, $M$ and $N$ are finitely generated $G$ modules such that $M \otimes _\Z \Q$ and $N \otimes _\Z \Q$ are isomorphic as $G$ modules. If either $h(M)$ or $h(N)$ is defined then so is other and they are equal.
\end{lem}













\chapter{Local Class Field Theory}
Standard reference for this topic is \cite{milneCFT}.
\section{Unramified Extensions}\label{sec1}
\begin{notation} 
We set $G= \text{Gal} (L/K)$ and $L/K$ finite and unramified extension of local field $K$. As usual, $\pi$ is uniformizer of $K$ and $U_L$ its units.
    
\end{notation} 
\begin{prop}
$L/K$ is unramified then $H^r(G,U_L)=0$.
\end{prop}
We know $ L^\times =U_L \pi ^ \mathbb  Z$ so $$H^r(G, L^\times ) \simeq H^r(G, U_L) \oplus H^r(G, \Z)$$
As $L/K$ is unramified it is cyclic and thus, as Tate cohomology groups will be 2 periodic, it suffices to show that $\hat H ^0 (G, U_L) =0$.








\section{The invarint map}
Consider the short exact sequence $$ 0 \rightarrow \Z \rightarrow \Q \rightarrow \Q /\Z \rightarrow 0$$  with trivial $G$-action on all three modules. \newline
Let $G$ be the galois group of arbitrary unramified extension $L/K$. 
$$H^1(G, \Q /\Z ) \xrightarrow{\delta} H^2(G,\Z)$$
$\text{Frob}_{L/K}$ be element of $G$ such that $\text{Frob}_{L/K}\big |_{E}=\text{Frob}_{E/K}$ for some finite subextension.
\begin{defi}[The invariant map]
Define the invariant map $$\text{inv}_{L/K} : H^2(G,L^\times) \rightarrow \Q/\Z $$ by following composition 
$$ H^2 (G, L^\times ) \xrightarrow{\text{ord} _L} H^2 (G,\Z) \xrightarrow{\delta ^{-1}} H^1(G, \Q/\Z)\xrightarrow{f\mapsto f(\text{Frob}_{L/K})} \Q/\Z $$
\end{defi}
\begin{notation}
    $K^{unr}$ is maximal unramified extension in $K^{sep}$.
\end{notation}
We have a unique isomorphism 
$$ \text{inv}_K : H^2(K^{unr}/K) \xrightarrow{\sim} \Q/\Z$$ 
such that for finite extension $L/K$, 
$$\text{Inf}:H^2(L/K) \rightarrow H^2(K^{unr}/K)$$ induces isomorphism $$\text{inv}_{L/K}: H^2(L/K) \xrightarrow{\sim} \frac{1}{[L:K]}\Z/\Z$$
We need to extend this map to ramified extensions. 
\begin{thm}
$L/K$ be a finte extension inside $K^{sep}$. then 
$$\begin{tikzcd}
H^2(K^{unr}/K) \arrow[d, "\text{inv}_K"] \arrow[r, "\text{Res}"] & H^2(K^{unr}/L) \arrow[d, "\text{inv}_L"] \\
\Q/\Z \arrow[r, "{[L:K]}"]                                       & \Q /\Z               
\end{tikzcd}$$
The $\text{Res}$ map is given by compatible maps $\tau \mapsto \tau \big |_{K^{unr}}: \gal(L^{unr}/L) \hookrightarrow \gal (K^{unr}/K)$ and ${K^{unr }}^\times \hookrightarrow {L^{unr}}^\times$.
\end{thm}
\begin{thm}[Class Field Axiom]
Let $L/K$ be cyclic of order $n$ then $$|H^r(G,L^\times )| = 
\begin{cases}
n \quad n \ \text{is even} \\
1 \quad n \ \text{is odd}
\end{cases}$$
\end{thm}

\section{Commutativity}
Commutativity of 
$$\begin{tikzcd}
L^\times/N_{F/L}(F^\times) \arrow[d, "N_{L/K}"] \arrow[r, "\phi_{F/L}"]  & \gal(F/L)^{ab} \arrow[d ] \arrow[r, equal]  &   {H/[H,H]} \arrow[d] \\
K^\times/N_{F/K}(F^\times) \arrow[r, "\phi _{F/K}"]& \gal(F/K)^{ab} \arrow[r, equal] &   {G/[G,G]}
\end{tikzcd}$$
Consider 
$$\begin{tikzcd}
0 \arrow[r] & I_G \arrow[r] \arrow[d, hookleftarrow] & \Z [G] \arrow[r] \arrow[d, hookleftarrow] & \Z \arrow[d, equal] \arrow[r]& 0 \\
0 \arrow[r] & I_H \arrow[r] &\Z [H] \arrow[r] & \Z \arrow[r] &0
\end{tikzcd}$$




\chapter{Central Simple Algebras}

\chapter{Lubin-Tate formal group laws}
\section{Power Series}
\begin{lem}
\begin{enumerate}
    \item For all power series $f\in A[[T]]$ and $g,h \in TA[[T]]$ we have $$f\circ (g\circ h)=(f\circ g) \circ h.$$
    \item Let $f= \sum _{n=1}^\infty $ then there exists $g$ such that $f \circ g=T$ if and only if $a_1$ is unit in which case the $g$ is unique and we have $g \circ f =T$ as well.
\end{enumerate}
\end{lem}

\section{Formal group laws}
\begin{defi}[Formal group law]
        Let $A$ be a commutative ring a formal group law $F$ is power series $\in A[[X,Y]]$ such that 
        \begin{enumerate}
        \item $F(X,Y)=X+Y + (\deg \geq 2)$ 
        \item $F(X, F(Y,Z))=F(F(X,Y),Z)$
        \item There exists unique $i_F(X) \in XA[[X]]$ such that $F(X,i_F(X))=0$.
        \item $F(X,Y)=F(Y,X)$
        \end{enumerate}
\end{defi}
		\begin{defi}Homomorphism of formal group laws
		$F\rightarrow G$ is a power series $h\in TA[[T]]$ such that $$h(F(X,Y))=G(h(X),h(Y))$$
		\end{defi}
\begin{notation}
Let $A$ be ring of integers of non-archimedean local field $K$, $\pi$ be prime element of $A$.

\end{notation}	
\begin{defi}
Let $\mathcal F_\pi$ be set of $f(X) \in A[[X]]$ such that 
\begin{enumerate}
    \item $f(X)= \pi X + (\deg \geq 2)$ 
    \item $f(X)=X^q \mod \pi$.
\end{enumerate}
\end{defi}

\begin{lem}\label{ltmainlemma}
Let $f,g \in \mathcal F _\pi$ and $\phi _1(X_1, \dots, X_n)$ linear form with coefficients in $A$. There exists unique $\phi \in A[[X_1, \dots ,X_n]]$ such that $$\phi (X_1, \dots ,X_n) =\phi _1 + (\deg \geq 2)$$ and $$f(\phi (X_1, \dots , X_n))=\phi (g(X_1), \dots , g(X_n)).$$
\end{lem}
\begin{proof}
Inductively prove that there exists unique polynomial $\phi _r(X_1, \dots , X_n)$ of degree $r$ such that  $$\phi _r (X_1, \dots, X_n) = \phi _1 + (\deg \geq 2)$$ and $$ f(\phi _r (X_1, \dots , X_n))=\phi _r(g(X_1), \dots , g(X_n)) + (\deg \geq r+1).$$
Define $\phi =\phi _r + (\deg \geq r+1)$. 
\end{proof}
According to the lemma there is unique power series $F_f(X,Y)$ such that $$F_f(X,Y)=X +Y + (\deg \geq 2)$$ and $$f(F_f(X,Y)) =F_f(f(X),f(Y)).$$ 
We can check that $F_f$ is a formal group law. \\
Such a group law, $F_f$, is called Lubin-Tate formal group law. 
\begin{defi}
Using \cref{ltmainlemma}, for $f,g \in \mathcal F _\pi$ and $a\in A$, define $[a]_{g,f}$ to be the unique power series in $A[[T]]$ such that $$[a]_{g,f}(T) =aT + (\deg \geq 2)$$ and $$g \circ [a]_{g,f}=[a]_{g,f}\circ f.$$
\end{defi}
Such $[a]_{g,f}$ is homomorphism $F_f \rightarrow F_g.$ 
\begin{lem}
We have $$ [a+b]_{g,f}=[a]_{g,f}+_{F_g} [b]_{g,f}$$ and $$ [ab]_{h,f} =[a]_{h,g}\circ [b]_{g,f}.$$
\end{lem}
\begin{rem}
Thus $[1]_{g,f}$ is isomorphism $F_f \simeq F_g$ and so is any $[u]_{g,f}$ for any $u \in A^\times$.
\end{rem}
\begin{rem}
$[\pi]_f =f $
\end{rem}
\begin{rem}
$ a\mapsto [a]_f$ gives injection $A \hookrightarrow \text{End} (F_f)$.
\end{rem}
\begin{defi}
$$\Lambda _f = \{ \alpha \in K^{al} \ | \ |\alpha | <1\} $$
$$ \alpha +_{\Lambda _f} \beta = \alpha +_{F_f} \beta $$
$$ a * \alpha =[a]_f (\alpha )$$
$\Lambda _n$ be submodule of $\Lambda _f$ killed by $[\pi ] _f^n$. 
\end{defi}
The $\Lambda _n$ is set of all roots of $f^{(n)}$ in $K^{al}$. We note that if $g$ is another element of $\mathcal F _ \pi$ then $A$-isomorphism $F_f \rightarrow F_g$ induces $A$-isomorphism $\Lambda _f \rightarrow \Lambda _g$.
\begin{thm} Let $K_{\pi, n}=K[\Lambda _n]$ then
\begin{enumerate}
    \item For all $n \geq 1$, $K_{\pi ,n }/K$ is totally ramified of degree $(q-1)(q^{n-1})$
    \item The action of $A$ on $\Lambda _n$ defines isomorphism $$(A/ \mathfrak m ^n)^\times \rightarrow \gal (K_{\pi,n}/K).$$
    \item For each $n$, $\pi \in N(K^\times_{\pi,n})$.
\end{enumerate}
\end{thm}
\begin{proof}
We let $f(T) = \pi T + T^q$. Choose $\pi _1$ to be the non zero root of $f(T)$ and then $\pi _n$ as root of $f(T) - \pi _{n-1}$. \\
$K[\pi _n] (\subset K[\Lambda _n])$ is totally ramified of degree $q^{n-1} (q-1)$.
$$ K[\Lambda _n] \supset K[\pi _n] \stackrel{q}{\supset} K[\pi _{n-1}] \stackrel{q}{\supset} \dots \stackrel{1}{\supset} K[\pi _1] \stackrel{q-1}{\supset} K$$ 
\end{proof}
\begin{prop}
$A$-module $\Lambda _n$ is isomorphic to $A/(\pi)^n$. $$\text{End}_A(\Lambda _n) \cong A/(pi)^n$$ and $$ \text{Aut}_A (\Lambda _n) \cong (A/(\pi)^n)^\times .$$
\end{prop}
\begin{proof}
The $\Lambda _f$ is a module over a principal ideal domain - $A$. \\
The polynomial $f=\pi T +T^q$ has $q$ distinct roots. So $\Lambda _1$ has $q$ elements. So $\Lambda _1 \cong A/(\pi)$. \\ For $\alpha \in K^{al} , |\alpha | <1$, $f(T) - \alpha $ has a root in $\Lambda _\alpha$ so $f : \Lambda _f \rightarrow \Lambda _f $ is surjection. \\
Use induction to prove that $\Lambda _n \cong A/(\pi )^n$.
Note that $$ 0 \rightarrow \Lambda _1 \rightarrow \Lambda _n \xrightarrow{f} \Lambda _{n-1} \rightarrow 0.$$
So $\Lambda _n$ has $q^n$ elements and if $\Lambda _n$ is not cyclic then neither is $\Lambda _1$.  Action of $A$ on $\Lambda _n$ induces $$A/(\pi)^n \xrightarrow{\sim} \text{End}_A(\Lambda _n).$$
\begin{lem}
Let $G = \gal (L/K), F \in \mathcal O_K [[X_1, \dots , X_n]]$ and $\alpha _1, \dots , \alpha _n \in \mathfrak m _L$
\end{lem}
So $\gal (K[\Lambda _n]/K) $ acts on $\Lambda _n$ s $A$-module isomorphism. So $\gal (K[\Lambda _n]/K)\subset (A/(\pi)^n)^\times$. So $$(q-1)q^{n-1} \geq |\gal(K[\Lambda _n]/K)|=[K[\Lambda _n]:K] \geq [K[\pi _n]:K]=(q-1)q^{n-1}$$
Also $K[\Lambda _n ]=K[\pi _n]$. \\
Define $f^{[n]}=\Big ( \frac{f}{T} \Big )\circ f^{n-1}$ $$f^{[n]}(T) = \pi + \dots + T^{(q-1)q^{n-1}}$$ $$f^{[n]} (\pi _n) =f^{[n-1]} (\pi _{n-1}) = \dots =0$$
So $f^{[n]}$ must be minimal polynomial of $f^{[n]}$.\\
$\pi \in N$.
\end{proof}

\begin{defi}
$$K_{\pi,n}=K[\Lambda _n]$$ $$ K_\pi = \cup K_{\pi ,n}$$ 
\end{defi}
We saw $$ (A/ \mathfrak m^n ) ^\times \simeq \gal (K_{\pi,n}/K) $$ $$A^\times \simeq \gal(K_\pi /K)$$
\begin{defi}[Local Artin Map]
Define $$ \phi _\pi : K^\times \rightarrow \gal ((K_\pi . K^{un})/K)$$ as follows: \\
for $a \in K^\times ,a=u\pi ^m$ \\
$\phi _\pi (a)$ acts on $K^{un}$ as $\text{Frob} ^m$ and \\
$\phi _\pi (a) (\lambda )= [u^{-1}]_f(\lambda) $ for all $lambda \in \cup \Lambda _n$.
\end{defi}
We will prove following main theorem:
\begin{thm}
For local field $K$ and a uniformizer $\pi$ of $K$, $K_\pi . K^{un}$ and $\phi _\pi$ are independent of the choice of the uniformizer.
\end{thm}
\begin{notation}
Denote valuation ring of $\hat {K^{unr}}$ by $B$ and $\sigma$ be extension of $\text{Frob} \in K^{unr}/K$ to the completion $\hat {K^{unr}}$. For $\theta (T) = \sum _{i=0} ^{\infty} b_i T^i \in B[[T]]$ define $(\sigma \theta )(T)$ to be the power series $\sum _{i=0}^\infty \sigma (b_i)T^i$. 
\end{notation}
\begin{prop}
Let $F_f,F_g$ be formal group laws defined by $f \in \mathcal F _\pi$ and $g\in \mathcal F _{\varpi}$ where $\pi $ and $\varpi = u \pi$ are primes of $K$. Then $F_f$ and $F_g$ are $A$-isomorphic over $B$. More precisely, there exists $\epsilon \in B^ \times $ such that $\frac{\sigma (\epsilon)}{\epsilon}=u$ and a power series $\theta (T) \in B[[T]]$ such that \begin{enumerate}
    \item $\theta = \epsilon T + (\deg \geq 2)$
    \item $\sigma \theta = \theta \circ [u]_f$
    \item $\theta (F_f(X,Y)) = F_g(\theta (X), \theta (Y)) $
    \item $\theta \circ [a]_f=[a]_g\circ \theta$.
\end{enumerate}The last two conditions imply that $f$ is homomorphism commuting with action of $A$ and the 
the first condition implies that $\theta$ is isomorphism.
\end{prop}
\begin{proof}
\begin{lem}
The maps $b\mapsto \sigma b -b:B \rightarrow B$ and  $b\mapsto \frac{\sigma b}{b}:B^\times \rightarrow B^\times $  are surjective with kernels $A$ and $ A^\x$.
\end{lem}
\begin{proof}
Let $R$ be valuation ring of $K^\un$ with maximal ideal  $\mathfrak n$. Then $R$ is discrete valuation ring and $\varprojlim R/\mathfrak n ^n = B$. We prove $$0 \rightarrow A/\mathfrak m ^n \rightarrow R/\mathfrak n^n \xrightarrow{\sigma -1} R / \mathfrak n^n \rightarrow 0$$ is exact using induction. So for $n=1$ the sequence becomes the familiar $$ 0 \rightarrow k \rightarrow \overline k \xrightarrow{x \mapsto x^q-x} \overline k \rightarrow 0.$$
Here $\overline k$ is algebraic closure of $k$. For induction refer to the following commutative diagram 
\[
\begin{tikzcd}
0 \arrow[r] & R/\mathfrak n \arrow[r] \arrow[d , "\sigma -1"] & R/\mathfrak n^n \arrow[r] \arrow[d, "\sigma -1"] & R/\mathfrak n^{n-1} \arrow[r] \arrow[d, "\sigma -1"] & 0 \\
0 \arrow[r] & R/\mathfrak n \arrow[r] & R/ \mathfrak n ^n \arrow[r] & R/\mathfrak n ^{n-1} \arrow[r] & 0
\end{tikzcd}
\]
From snake lemma $\sigma -1: R/\mathfrak n^n \rightarrow R/ n ^n$ is surjective and has $q^n$ elements. Its kernel contains $A/\mathfrak m ^n$ whose size is $q^n$ so must be the full kernel. Taking inverse limit using \cref{inverselimitexact} we get the lemma. The proof for the second map is similar.
\end{proof}
This lemma gives existence of $\epsilon$. \\
We should define $\theta$ inductively so that  $\theta _1(T) = \epsilon T$ \[\theta _r(T) = \theta _{r-1} (T) + b T^r \quad b \in B\]
\[ \sigma \theta _r = \theta _r \circ [u]_f + ( \deg \geq r+1) \]
Note that our choice of $\epsilon $ ensures that the second of the above condition is satisfied. Suppose we have found $\theta _r$. If we let $b= a\epsilon ^{r+1}, \ a\in B$ from the first and the second conditions above we get that we want $a$ to be solution of \[ (\sigma a -a )(\epsilon u) ^{r+1} =c\] where $c$ is the coefficient of $T^{r+1}$ in $\theta _r \circ [u]_f - \sigma \theta _r$. Again, the lemma proved guarantees such $a$. \\
Define $h= \sigma \theta \circ f\circ \theta ^{-1}$. We will show that $h \in \mathcal F _\varpi$. 
$h(T) \in A[[T]]$ as $\sigma h = \sigma \theta \circ [u]_f \circ f \circ \sigma \theta ^{-1} = h$. Also note that \[h(T)= \sigma \epsilon . \pi. \epsilon ^{-1} T + \dots = \varpi T + (\deg \geq 2).\] Also,
\[ h(T)= \sigma \theta \circ f (\theta ^{-1}) (T) = \sigma \theta (\theta ^{-1}(T))^q = \sigma \theta (\sigma \theta ^{-1} (T^q)) =T^q \mod \mathfrak m\] 
Let $\theta ' = [1]_{g,h} \circ \theta$. Then one verifies that this $\theta '$ is the one promised in the proposition.
 \end{proof}
\begin{thm}
$K_\pi . K^\un$ is independent of $\pi$.
\end{thm}
\begin{proof}From above proposition,
\[\sigma \theta \circ [\pi]_f=[\varpi ]_g \circ \theta \] so $\theta$ gives a bijection $ \Lambda _{f,1} \rightarrow \Lambda _{g,1}$. \\
$\hat K ^\un (\Lambda _{f,1})= \hat K ^\un (\Lambda _{g,1})$ as \[ \hat K ^\un (\Lambda _{g,1})= \hat K ^\un (\theta (\Lambda _{f,1})) \subset \hat K ^\un (\Lambda _{f,1} = \hat K ^\un (\theta ^{-1} (\Lambda _{g,1})) \subset \hat K ^\un (\Lambda _{g,1}).\]
We will prove that for any two prime elements $\hat K^ \un (\Lambda _{f,1}) \cap K^ \al = K^ \un (\Lambda _{f,1}) $. This is so because if we have a subextension $K\subset E \subset K^\al$ then $\gal (K^\al/E)$ fixes the closure of $E$ by continuity thus $ \hat E \cap K^\al \subset E$. So $K ^\un (\Lambda _{f,1}) = K ^\un (\Lambda _{g,1})$. Similarly $K ^\un (\Lambda _{f,n})= K^\un (\Lambda _{g,n})$ which implies $K^ \un K _\pi= K^ \un K_\varpi$.
\end{proof}
\begin{thm}
$\phi_\pi$ is independent of $\pi$.
\end{thm}
\begin{proof}
We show that for any two prime elements \[\phi _\pi(\varpi )=\phi _\varpi (\varpi )\]
On $K ^\un$ both act as Frobenius automorphism, so we only need to check $K_\varpi$. \\
Note $\phi _\varpi (\varpi)$ is identity on $K_\varpi$. \\
Let $\theta$ be isomorphism $F_f \rightarrow F_g$ over $\hat K ^\un$ as in the above proposition. It induces isomorphism $ \Lambda _{f,n} \rightarrow \Lambda _{g,n}$ so we need to prove that for all $\lambda \in \Lambda _{f,n}$ \[ \phi _\pi (\varpi ) (\theta (\lambda )) = \theta (\lambda )\]
which follows as \[ \phi _\pi (\varpi )(\theta (\lambda ))= \\phi _\pi (\pi ) (\theta)(\phi _\pi (u) \lambda )=(\phi_\pi (\pi) \theta )[u^{-1}]_f (\lambda)=\theta (\lambda) \] Note we got the second last inequality again by using the above proposition.
\end{proof}
\begin{thm}
\[K^\ab =K _\pi . K^\un \]
\end{thm}
\begin{proof}
Let $U_{n,m}=(1 + \mathfrak m ^n). <\pi ^m>$ and $K_{n,m}=K_{\pi,n}.K_m$ where $K_m $ is degree $m$ unramified extension. \\
For all $a\in U_{n,m}$ we know $ \phi _\pi (a) \big | _{K_{n,m}}=1$. 
\begin{lem}
For all $a\in K^\x$, $\phi (a) $ and $\phi _\pi (a)$ agree on subfield of $K_{\pi}.K^\un $.
\end{lem}
\begin{proof}
See \cite{milneCFT} p. 26, 1.14.
\end{proof}
From the lemma $U_{n,m} \subset N(K^\x _{n,m})$ and one can easily see that $[K^\x :U_{n,m}]=[K_{m,n}:K]$. Thus $U_{n,m}= N(K^\x _{n,m})$. \\
For finite index abelian extension $L$ of $K$, $N(L^\x)$ has finite index in $K^\x$ and thus is open. So for some $m,n$, $U _{n,m} \subset N(L^\x)$. This gives us $L \subset K_{n,m}$.
\end{proof}
\begin{cor}
Every open subgroup of finite index contains $U_{n,m}$ so a norm group, namely $N(K^\x _{n,m})$. So every open subgroup is a norm group.  
\end{cor}

\chapter{Global Class Field Theory}
\section{Restricted topological product}
\begin{defi}
Let $I$ be an indexing set and $X_i$ topological space for all $i \in I$ let $A_i \subset X_i$ be open subsets. Then the restricted topological product of $X_i$s relative to open subsets $A_i$ is the set \[ \prod _{i\in I} (X_i,A_i) = \Big \{(x_i)_{i\in I} \in \prod _{i\in I} X_i \ \Big | \ x_i \in A_i \text{ for all but finitely many i} \Big  \}\]
We gave topology on this space such that the basic open sets are \[ \{(x_i)_{i\in I} \ | \ x_i \in U_i \text{ for } i \in J \text{ and } x\in A_i \text{ for } i \in I \}\] 
where $J \subset I$ is finite and $U_i$ is open subset of $X_i$.
\end{defi}
\begin{lem} If for all $i$, $X_i$ is Hausdorff and $A_i$ is open subset of $X_i$ then $\prod _{i\in I} (X_i,A_i)$ is Hausdorff.
\end{lem}
\begin{lem}
If for aall $i\in I$, $G_i$ is locally compact, Hausdorff topological group or ring and $K_i$ is open compact subgroup or subring then $\prod _{i\in I} (G_i,K_i)$ is locally compact, Hausdorff topological group or ring.
\end{lem}	
\section{Adeles}
Let $K$ denote a global field. 
\begin{defi}
We define the adele ring $\A_K$ as \[ \A _K = \prod _{v \in V_K} (K_v,\mathcal O _v)\]
where $V_K$ is the set of all places of $K$ that is the set of equivalence classes of valuations. 
\end{defi}
\begin{notation}
$K_v$ is completion of $K$ w.r.t. $v$ (so it is a local field) and $\Ol _v$ is valuation ring of $v$. 
\end{notation}
\begin{defi}
Define the diagonal embedding map $\delta _K : K \rightarrow \A _K $ as $a \mapsto (a) _v \in V_K$.
\end{defi}
\begin{prop}
The diagonal embedding has discrete image.
\end{prop}
\begin{proof} It is enough to prove that there is a open set containing $0$ and no other element of $K$ embedded inside $\A _K$.
Consider the set \[ U =  \{ (\alpha _v) \  | \ ||\alpha _\infty|| _\infty <1 \text{ and } || \alpha  _v|| _v \leq 1 \text{ for all } v < \infty \} \] 
Recall that by product formula for $K$, $ \prod _{v} ||\alpha _v || _v =1$ for any non zero $\alpha $ so $U \cap K = \{ 0 \}$.  
\end{proof}
\section{Ideles}
\begin{defi}
We define idele group as \[ \I _K = \prod _{v\in V_K} (K^\x _v , \Ol ^ \x _v)\]
\end{defi}
\begin{defi}
We define continuous content homomorphism $c_K : \I _K \rightarrow \R _{>0}$ as \[c_K (\alpha ) = \prod _{v \in V_K} \| \alpha \| _v\]
Define $\I ^1 _K$ as kenrel of $c_K$.
\end{defi}

\begin{rem}
$K^\x \subset \I _K ^1$ due to the product formula.
\end{rem}
\begin{lem}
The topology on $\I ^1 _K$ form $\I _K$ agrees with subspace topology from $\A _K$.
\end{lem}
\begin{proof}
Let $x \in \I ^1 _K $ and open neighbourhood $U \subset \I _K$ of $x$ contains \[ u \ | \ \|u-x\|_v < \epsilon \text{ for } v\in S \text{ and } \| u \|_v=1 \text{ otherwise} \}\]
where $S$ is finite subset of $V_K$. \\
Consider open set $V$ of $\A _K$. \[ \{ u \ | \ \|u-x\|_v < \epsilon \text{ for } v\in S \text{ and } \| u \| _v \leq 1 \ \text{otherwise} \} \]
Observe that $\I ^1 _K \cap U = \I^1 _K \cap V$. This along with the observation that $\I  _K$ has finer topology than $\A _K$ gives the result.
\end{proof}
\begin{defi}
Define the idele class group as $\C_K = \I _K/K^\x$.
\end{defi}
\begin{prop}
Define map $\pi _K : \I _K \rightarrow I_K$ as $\pi _K (\alpha) = \prod _{v \in V_{K,f}} \mathfrak p _v ^{v(\alpha )}$. Here $V_{K,f}$ denotes the set of finite places of $K$ and $\mathfrak p _v$ denotes prime corresponding to finite place $v$. For a number field $K$, $\pi _K( \I^1 _K) = I_K$ and $\pi $ is continuous once we give $I_K$ discrete topology. 
\end{prop}
\begin{proof}
Consider idele of content $1$ which is $\pi _\mathfrak p$ at the place corresponding to $\mathfrak p$ and $\frac{1}{\|\pi _\mathfrak p \|_{\mathfrak p}^{\frac{1}{[K:\Q ]}}}$ at archimedean place and $1$ everywhere else.
\end{proof}
\section{Statements of Global Class Field Theory}
\begin{lem}
Let $L/K$ be finite extension of global fields and $\alpha \in \I _K$. Then $\phi _{L_w/K_v} (\alpha _v) =1$ for all places $w$ above place $v$ for all but finitely many $v$.
\end{lem}
\begin{lem}
Let $\alpha \in K_v^\x $ then \[\phi _{L_w/K_v}(\alpha ) \big | _L \in \gal (L/K)\] is independent of a place $w$ lying over $v$.
\end{lem}
\begin{proof}
We have canonical map \[G^\ab _{L_w/K_v} \rightarrow D_w \rightarrow G _{L/K} ^\ab \] where $D_w$ is the decomposition group at place $w$ over $v$. This map is independent of choice of $w$ since action by conjugation is trivial on $G^\ab _{L/K}$.
\end{proof}
\begin{defi}[Global reciprocity map]
For finite extension $L/K$ of global fields define \[ \Phi _{L/K} : \I _K \rightarrow \gal (L/K) \] as \[\Phi _{L/K} (\alpha ) = \prod _{v\in V_K} \phi _{L_w/K_v}(\alpha _v) \big |_L \] where $w$ is some place over $v$.  \\
As $\phi _{L_w/K_v}(\alpha _v)= \phi _{M_u/K_v}(\alpha _v) \big |_{L_w} $ for $L \subset M$ and $w \in V_L$ over $v$ and $u \in V_M$ over $w$, it follows that $\Phi _{L/K}(\alpha ) = \Phi _{M/K} (\alpha ) \big |_L$ so we may define \[ \Phi _K : \I _K \rightarrow G^\ab _K \] as $\Phi _K = \varprojlim \Phi _{L/K}$.
\end{defi}
\begin{thm}[Global Reciprocity] 
$L/K$ abelian extension then 
\begin{enumerate}
\item $\Phi _K (a) =1 $ for all $a \in K^\x$.
\item $\Phi _{L/K}$ is surjective with kernel $K^\x N_{L/K}(\I _L)$.

\end{enumerate}
where \[ N_{L/K} ((a_u)_u) = \big ( \prod _{u |v} N_{L_u/K_v} a_u ) _v .\]
We may also state this as \[ \C _K/N_{L/K}(\C _L) \xrightarrow{\sim} \gal (L/K) .\]
\end{thm}
\begin{thm}[Existence theorem for global class field theory]
The open subgroups of $\C _K$ of finite index are exactly the norm subgroups of finite abelian extension $L/K$.
\end{thm}
\begin{notation}
Henceforth, in this section, $K$ is a number field. 
\end{notation}

\begin{defi}
Define modulus $\mathfrak m $ by a formal product as $\mathfrak m = \mathfrak m_f \mathfrak m _{\infty} $. Here $\mathfrak m _f$ is formal product of finite places and $\mathfrak m _\infty $ of infinite.
\end{defi}

\begin{defi}[Artin map]
For finite abelian extension define Artin map \[\Psi _{L/K} ^\mathfrak m : I ^\mathfrak m _K \rightarrow \gal (L/K) \]
where $\mathfrak m$ is a modulus such that every place that ramifies divides $\mathfrak m$ and \\
 $I ^\mathfrak m _K$ is $\mathfrak m$-ideal group is a subgroup of ideal group generated by non zero prime ideals which do not divide $\mathfrak m$ such that \\
 $\Psi ^\mathfrak m _{L/K} (\mathfrak p)$ is Frobenius element at $\mathfrak p$ for $\mathfrak p$ which does not divide $\mathfrak m _f$.
 \end{defi}
 
\begin{defi} 
We define $a = ^* b$ if $a=b \mod \mathfrak m_f $ and $\frac{a}{b}$ in real embedding is $>0$. \\
Define \[K_{\mathfrak m ,1}= \{ a \in K^\x \ | \ a = ^* 1 \mod \mathfrak m\} \] \\
Let $P^ \mathfrak m _K$ is fractional ideals generated by elements of $K_{\mathfrak m, 1}$. \\
Ray class group $Cl_K ^\mathfrak m$ is defined as $I_K ^\mathfrak m / P_K ^\mathfrak m $. \\
The reciprocity map is $\Psi _K ^\mathfrak m : Cl_K ^\mathfrak m \rightarrow \gal (L/K)$.\\ 
Defning modulus is such that every prime that ramifies divides $\mathfrak m$ and $P^\mathfrak m _K \subset \ker \Psi ^\mathfrak m _{L/K}$. 
\end{defi}
 This map induces a map on ray class group when $\mathfrak m$ is defining modulus 
\section{First inequality}
\subsection{Cohomology of ideles}
\begin{prop}
Let $v$ be a place of $K$ and $u$ be a place of $L$ lying above it and $G=\gal(L/K)$ then \[\text{Ind}_{G_u}^G(L_u ^\x) \simeq \prod _{w|v} L^\x _w \] and \[ \text{Ind} _{G_u}^G(\Ol _u ^\x ) \simeq \prod _{w|v} \Ol _w ^\x \]
\end{prop}
\begin{proof}
We know $\sigma $ induces isomorphism $\sigma : L_u \rightarrow L_{\sigma (u)}$. Consider the restriction of \[ \Z [G] \otimes _{\Z[G_u]}L_u \rightarrow \prod _{w|v}L_w\]
induced by $(\sigma , \beta ) \mapsto \sigma (\beta )\in L_{\sigma (u)} .$
\end{proof}
\begin{defi}
Suppose $S$ is a finite set of places of $K$, define \[ \I _{L,S} = \prod _{v\in S} \prod _{w|v} L^\x _w \times \prod _{v\not \in S } \prod _{w|v} \Ol ^\x _w \]
\end{defi}

\begin{prop}
Let $S$ be a finite set of places containing the archimedean places and places that ramify in $L/K$, we have isomorphism for all $i \in \Z$ and a place $w|v$ 
\[ \hat H^i (G, \I _{K,S} ) \simeq \oplus _{v\in S} \hat H^i (G_w,L_w ^\x ) .\]
\end{prop}
\begin{proof}
By Shapiro's lemma (\Cref{shapiro}) \[ \hat H^i(G_u, L_u ^\x) \cong H^i(G, \prod _{w|v} L_w ^\x) \] and \[ \hat H^i(G_u, \Ol ^\x _u) \cong \hat H^i (G, \prod _{w|v} \Ol ^\x _w) \] for all $i \in \Z$. As Tate cohomology commutes with products, 
\[ \hat H^i (G, \I _{L,S} ) \cong \prod _{v\in S} \hat H^i \Big( G, \prod _{w|v}L^\x _w\Big) \x \prod _{v \not \in S} \hat H^i\Big(G, \prod _{w|v} \Ol ^\x _w\Big) \] which gives us the result using above observation and the fact that $\hat H^i (G_w, \Ol ^\x_w) =0$ for all $i \in \Z $.
\end{proof}
\begin{rem}
So $|\hat H^0 (G,\I _{L,S})| = \prod _{v\in S} |G_w| $ by Local class field theory. \\
By Hilbert $90$  we get $H^1(G, \I _{L,S})=0$ and 
by the invariant map of Local class field theory \[H^2(G, \I _{L,S}) = \bigoplus _{v\in S} \frac{1}{|G_w|}\Z /\Z \]
\end{rem}
\begin{rem}
As $\I _L = \varinjlim _S \I _{L,S}$ we get \[ \hat H^i(G, \I _L)\cong \oplus _{v\in V_K} \hat H^i (G_w, L_w ^\x) \]
so $H^1(G, \I _L) =0$ and \[H^2 (G, \I _L) \cong \bigoplus _{v\in V_K} \frac{1}{G_w} \Z /\Z \]
\end{rem}
\begin{defi}
Define ring of $S$-integers \[\Ol _{K,S} = \{ \alpha \in K \ | \ \alpha \in \Ol _v \text{ for all } v \text{ such that } v\not \in S \} .\]
\end{defi}
Recall Dirichlet's unit theorem. It says $\Ol _K^\x \simeq \Z ^{r_1(K ) + r_2 (K) -1 } \x \mu (K) $. Below we generalize it for ring of $S$-integers.
\begin{prop}
\[ \text{rank}_\Z \Ol ^\x _{K,S} = |S| -1.\]
\end{prop}
\begin{proof}
Consider the exact sequence 
\[1 \rightarrow \Ol _K ^\x \rightarrow \Ol _{K,S} ^\x \xrightarrow{\sum_{v\in S_f} v} \oplus _{v\in S_f} \Z \]
here $S_f$ is finite places in $S$.
We only need to show that there exists a $S$-unit with nonzero valuation at a given $v\in S_f$ and zero at all other places in $S_f$. Consider $\mathfrak p$ corresponding to $v$ some power of $\mathfrak p$ is principal due to finiteness of size of class group. Generator of this ideal is then the required $S$-unit.
\end{proof}
\begin{lem}\label{idelefield}
Suppose $S$ contains set of finite places generating the ideal class group of $\Ol _K$, then $\I _K = \I _{K,S} K ^\x$
\end{lem}
\begin{proof}
Note that $\I _{K,S}$ maps surjectively onto $Cl_K$. Let $\phi : \I _K \rightarrow Cl _K $ defined by $(\alpha )_v \mapsto \prod _{v \in V_f} \mathfrak p _v ^{v(\alpha )} $. Then $\ker \phi = \Ol . K^\x$ where $\Ol $ is set of local units at finite places and local multiplicative group at infinite places. Thus we have \[\I _K /\Ol . K^\x \cong Cl _K \quad \text {and } \quad \I _{K,S}. K^\x/ \Ol . K^\x \cong Cl_K \]  \[ (\I _K / \Ol . K^\x )/(\I _{K,S} . K^\x/\Ol . K^ \x ) \cong Cl_K /Cl_K  \cong 1 \] Thus $\I _K / \I _{K,S} K ^\x \cong 1 \implies \I _K =\I _{K,S} K^\x $
\end{proof}
\begin{thm}
Let $L/K$ be cyclic then $h(G,\C _L) =[L:K]$.
\end{thm}
\begin{proof}
Let $S$ contain ramified primes and finite places lying below primes generating the ideal class group of $\Ol _L$. 
Note that $\I _{K,S} \cap K^ \x = \Ol _{K,S} ^\x$. So from \cref{idelefield}
\[ \C _L \simeq \I _{L,S} / \Ol _{L,S} ^\x \]
and thus \[h(G, \C _L) = \frac{h(G, \I _{L,S})}{h(G,\Ol _{L,S}^\x)} \]
Let $S_L$ be set of places in $L$ over $S$ and $V$ be $\R$-vector space with elements in $S_L$ as basis. Then $G$ acts on $V$. Consider $\Z [G]$ submodule $A$ generated over $\Z$ by the basis. \[ A \cong \oplus _{v \in S} \oplus _{w|v} \Z _w \cong \oplus _{v \in S} \text{Ind} ^G _{G_w} (\Z) \] \[h(G,A) = \prod _ {v\in S} h(G_w, \Z ) =  \prod _{v \in S} [L_w : K_v]\]
Define $l_{L,S} : \Ol ^\x _{L,S} \rightarrow V $ such that $l_{L,S} (\beta ) = ( \log \| \beta \| _w ) _{w \in S_L } $. By \cref{coeffbound} we have that $\ker l_{L,S}$ is finite.
\newline
By product formula the image of $l_{K,S}$ is contained in the huperplane $V^0$ of elements that sum to $0$. 
\[ \text{rank } \Ol ^\x _{L,S} = \text{rank} \Ol _L^\x + \text{ number of finite places in } S_L \] 
So the image of $l_{L,S}$, $B^0$, is complete lattice in $V^0$. 
\newline
Let $x = (1) _{w \in S_L} \in V ^G$ and let \[ B = \Z x + B^0 \] Observe that $B$ is complete lattice in $V$. We have \[ 0 \rightarrow B^0 \rightarrow B \rightarrow \Z x \rightarrow 0\]  so $h(G,B)=h(G, B^0)$using \cref{herbrandtensorq}  $h(G,\Z ) = h(G, \Ol _{L,S}^\x).[L:K]$.  
\newline
Complete lattice in finite dimensional vector spaces are isomorphic upon tensor product with $\Q$. So $h(G,A) =h(G,B)$ that is $h(G, \Ol _{L,S} ^\x ) = \frac{1}{n} \prod _{v \in S} [L_w : K_v]$.
Thus we get the theorem.

\end{proof}

\subsection{Consequences}
\begin{cor}
Let $L/K$ finite abelian extension, $D$ a subgroup of $\I _K$ such that \begin{enumerate}
\item $D \subset N_{L/K} (\I _L)$
\item $K^\x D $ is dense in $\I _K$ 
\end{enumerate}
Then $L =K$.
\end{cor}
\begin{proof}
Assume $L/K$ is cyclic WLOG.
$N_{L/K}(\I _L)$ is open in $\I _K$ so $K^\x N_{L/K} (\I _L)$ is open, and therefore closed, and dense. So by first inequality $L=K$. 
\end{proof}



\begin{lem}
Let $S$ be any finite set of places such that it contains all the archimedean and ramified places. Then $\gal (L/K)$ is generated by $\text{Frob} (v)$ such that $v \not \in S$.
\end{lem}
\begin{cor}
If $L/K$ is non trivial abelian then there are infinitely many primes that do not split completely.
\end{cor}
\subsection{Chebotarev density theorem}
\begin{defi} Natural density is also called asymptotic density is probability of encountering desired number in $[1,n]$ as $n\rightarrow \infty$.
\end{defi}
Let $L/K$ be galois extension of number fields. Let $\big( \frac{L/K}{\mathfrak p} \big)$ be the conjugacy class of $\text{Frob}_{L/K} (\mathfrak p)$ for $\mathfrak p$ unramified in $L/K$. 
\begin{thm}[Chebotarev density theorem]
Let $A=\{ \mathfrak p \ | \ \big( \frac{L/K}{\mathfrak p} = \mathcal C \big) \}$. The density of $A$ is $\delta (A) = \frac{|\mathcal C| }{|G|} $.
\end{thm}
As corollary we get Dirichlet's theorem of infinitude of primes in arithmetic progression. To see this consider $L =\Q(\zeta _n) , K= \Q$, for positive integers such that $(a,n)=1$ apply the theorem to $\mathcal C = \{ \zeta _n \mapsto \zeta ^a _n \}$.







%% Reference styles of Books and research articles are different. Some examples are given below.
\bibliographystyle{plain}
\bibliography{thesis}
\end{document}
